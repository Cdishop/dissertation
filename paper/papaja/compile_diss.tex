\documentclass[english,,man]{apa6}
\usepackage{lmodern}
\usepackage{amssymb,amsmath}
\usepackage{ifxetex,ifluatex}
\usepackage{fixltx2e} % provides \textsubscript
\ifnum 0\ifxetex 1\fi\ifluatex 1\fi=0 % if pdftex
  \usepackage[T1]{fontenc}
  \usepackage[utf8]{inputenc}
\else % if luatex or xelatex
  \ifxetex
    \usepackage{mathspec}
  \else
    \usepackage{fontspec}
  \fi
  \defaultfontfeatures{Ligatures=TeX,Scale=MatchLowercase}
\fi
% use upquote if available, for straight quotes in verbatim environments
\IfFileExists{upquote.sty}{\usepackage{upquote}}{}
% use microtype if available
\IfFileExists{microtype.sty}{%
\usepackage{microtype}
\UseMicrotypeSet[protrusion]{basicmath} % disable protrusion for tt fonts
}{}
\usepackage{hyperref}
\hypersetup{unicode=true,
            pdftitle={Thinking Longitudinal: A Framework for Scientific Inferences with Temporal Data},
            pdfauthor={Christopher R. Dishop, Michael T. Braun, Goran Kuljanin, \& Richard P. DeShon},
            pdfkeywords={longitudinal inferences, between-unit, growth, trends, dynamics,
relationships over time, processes},
            pdfborder={0 0 0},
            breaklinks=true}
\urlstyle{same}  % don't use monospace font for urls
\ifnum 0\ifxetex 1\fi\ifluatex 1\fi=0 % if pdftex
  \usepackage[shorthands=off,main=english]{babel}
\else
  \usepackage{polyglossia}
  \setmainlanguage[]{english}
\fi
\usepackage{graphicx,grffile}
\makeatletter
\def\maxwidth{\ifdim\Gin@nat@width>\linewidth\linewidth\else\Gin@nat@width\fi}
\def\maxheight{\ifdim\Gin@nat@height>\textheight\textheight\else\Gin@nat@height\fi}
\makeatother
% Scale images if necessary, so that they will not overflow the page
% margins by default, and it is still possible to overwrite the defaults
% using explicit options in \includegraphics[width, height, ...]{}
\setkeys{Gin}{width=\maxwidth,height=\maxheight,keepaspectratio}
\IfFileExists{parskip.sty}{%
\usepackage{parskip}
}{% else
\setlength{\parindent}{0pt}
\setlength{\parskip}{6pt plus 2pt minus 1pt}
}
\setlength{\emergencystretch}{3em}  % prevent overfull lines
\providecommand{\tightlist}{%
  \setlength{\itemsep}{0pt}\setlength{\parskip}{0pt}}
\setcounter{secnumdepth}{0}
% Redefines (sub)paragraphs to behave more like sections
\ifx\paragraph\undefined\else
\let\oldparagraph\paragraph
\renewcommand{\paragraph}[1]{\oldparagraph{#1}\mbox{}}
\fi
\ifx\subparagraph\undefined\else
\let\oldsubparagraph\subparagraph
\renewcommand{\subparagraph}[1]{\oldsubparagraph{#1}\mbox{}}
\fi

%%% Use protect on footnotes to avoid problems with footnotes in titles
\let\rmarkdownfootnote\footnote%
\def\footnote{\protect\rmarkdownfootnote}


  \title{Thinking Longitudinal: A Framework for Scientific Inferences with
Temporal Data}
    \author{Christopher R. Dishop\textsuperscript{1}, Michael T.
Braun\textsuperscript{2}, Goran Kuljanin\textsuperscript{3}, \& Richard
P. DeShon\textsuperscript{1}}
    \date{}
  
\shorttitle{LONGITUDINAL INFERENCES}
\affiliation{
\vspace{0.5cm}
\textsuperscript{1} Michigan State University\\\textsuperscript{2} University of South Florida\\\textsuperscript{3} DePaul University}
\keywords{longitudinal inferences, between-unit, growth, trends, dynamics, relationships over time, processes}
\usepackage{csquotes}
\usepackage{upgreek}
\captionsetup{font=singlespacing,justification=justified}

\usepackage{longtable}
\usepackage{lscape}
\usepackage{multirow}
\usepackage{tabularx}
\usepackage[flushleft]{threeparttable}
\usepackage{threeparttablex}

\newenvironment{lltable}{\begin{landscape}\begin{center}\begin{ThreePartTable}}{\end{ThreePartTable}\end{center}\end{landscape}}

\makeatletter
\newcommand\LastLTentrywidth{1em}
\newlength\longtablewidth
\setlength{\longtablewidth}{1in}
\newcommand{\getlongtablewidth}{\begingroup \ifcsname LT@\roman{LT@tables}\endcsname \global\longtablewidth=0pt \renewcommand{\LT@entry}[2]{\global\advance\longtablewidth by ##2\relax\gdef\LastLTentrywidth{##2}}\@nameuse{LT@\roman{LT@tables}} \fi \endgroup}


\DeclareDelayedFloatFlavor{ThreePartTable}{table}
\DeclareDelayedFloatFlavor{lltable}{table}
\DeclareDelayedFloatFlavor*{longtable}{table}
\makeatletter
\renewcommand{\efloat@iwrite}[1]{\immediate\expandafter\protected@write\csname efloat@post#1\endcsname{}}
\makeatother
\usepackage{lineno}

\linenumbers

\authornote{This article is currently in press for the
Handbook of Temporal Dynamic Organizational Behavior. It can be cited
as:

Dishop, C. R., Braun, M. T., Kuljanin, G., \& DeShon, R. P. (In press).
Thinking Longitudinal: A Framework for Scientific Inferences with
Temporal Data. In Griep, Y. Hansen, S. D., Vantilborgh, T., \& Hofmans,
J. (Ed.), Handbook of Temporal Dynamic Organizational Behavior. Edward
Elgar Publishing.

Correspondence concerning this article should be addressed to
Christopher R. Dishop, 316 Physics Road, Psychology Building, Room 348,
East Lansing, MI, 48823. E-mail:
\href{mailto:dishopch@msu.edu}{\nolinkurl{dishopch@msu.edu}}}

\abstract{
In this manuscript we explore how to think about patterns contained in
longitudinal or panel data structures. Organizational scientists
recognize that psychological phenomena and processes unfold over time
and, to better understand them, organizational researchers increasingly
work with longitudinal data and explore inferences within those data
structures. Longitudinal inferences may focus on any number of
fundamental patterns, including construct trajectories, relationships
between constructs, or dynamics. Although the diversity of longitudinal
inferences provides a wide foundation for garnering knowledge in any
given area, it also makes it difficult for researchers to know the set
of inferences they may explore with longitudinal data, which statistical
models to use given their questions, and how to situate their specific
inquiries within the broader set of longitudinal inferences. Moreover,
the diversity of statistical models that can be applied to longitudinal
data requires that researchers understand how one inference category
differs from another.


}

\usepackage{amsthm}
\newtheorem{theorem}{Theorem}[section]
\newtheorem{lemma}{Lemma}[section]
\theoremstyle{definition}
\newtheorem{definition}{Definition}[section]
\newtheorem{corollary}{Corollary}[section]
\newtheorem{proposition}{Proposition}[section]
\theoremstyle{definition}
\newtheorem{example}{Example}[section]
\theoremstyle{definition}
\newtheorem{exercise}{Exercise}[section]
\theoremstyle{remark}
\newtheorem*{remark}{Remark}
\newtheorem*{solution}{Solution}
\begin{document}
\maketitle

Imagine a close colleague that frequently agrees to volunteer for
additional work when asked to do so. What causes her to act this way?
Our intuition says that the cause must be something unique about her, a
motive, personality trait, disposition, or her momentary enthusiasm. So
it is with our research: the literature on correlates of why someone
responds with help has focused almost exclusively on individual
characteristics, such as affect, motives, attributions, justice or
leadership perceptions, personality, and vigor. But this emphasis
contradicts what we know about random processes, namely that long-run
streaks of behavior can be by byproducts of chance. Because chance
explanations have not been ruled out, statements about the necessity for
organizations to monitor, evaluate, and influence individual
characteristics to improve employee helping may be overblown. Moreover,
a manager who reads this literature and then assumes that individual
characteristics cause helping is more likely to falsely attribute good
character to her employees when she witnesses it, leading to performance
evaluations and reward recommendations that are, perhaps, biased. The
purpose of this paper is to find evidence of randomness in the requests
that employees receive asking them for assistance. If we identify
chance, then researchers, managers, and consultants must account for it
if they truly want to know whether something unique about the
individual, rather than something random about the situation, led to
exceptional, long-run helping. In the organizational literature, helping
or providing assistance to colleagues is referred to as organizational
citizenship.

Organizational citizenship behaviors (OCBs), or cooperative acts such as
assisting others, volunteering for additional work, or speaking highly
of the company, are increasingly emphasized in the organizational
sciences (HANDBOOK; dalal). Leaders put OCBs on equal footing to task
performance when asked about the merits of different behaviors within
their teams (Podsakoff, MacKenzie, \& Podsakoff, 2018), and researchers
have gone so far as to describe OCBs as the key social aspect driving
organizational success (Bateman \& Organ, 1983). Researchers, as well as
consultants, managers, and employees, are interested in knowing why
people differ on this behavior, and in particular why someone might have
sustained, superior levels of OCBs over time.

Employees that exhibit sustained, high-levels of OCBs are labeled
\enquote{extra-milers} or \enquote{good citizens} in the literature (Li,
Zhao, Walter, Zhang, \& Yu, 2015; Methot, Lepak, Shipp, \& Boswell,
2017), and researchers have identified a number of predictors of this
behavior -- many of which are individual characteristics. These include
prosocial motivation and personality (Grant, 2008; Penner, Midili, \&
Kegelmeyer, 1997), impression management motives (Grant \& Mayer, 2009),
one's propensity to be concerned for others (Meglino \& Korsgaard,
2004), job satisfaction, perceived fairness, and organizational
commitment (Organ \& Ryan, 1995), perceptions of trust (HANDBOOK;
moorman, fit (HANDBOOK; kristof brown), leader fairness (HANDBOOK;
piccolo), and interaction quality with colleagues (Bolino, Hsiung,
Harvey, \& LePine, 2015), how employees appraise pressures to perform
and goals (Mitchell, Greenbaum, Vogel, Mawritz, \& Keating, 2019), their
level of engagement and mindfullness (Hafenbrack et al., 2019; Wang,
Law, Zhang, Li, \& Liang, 2019), and their perceptions of ostracism
(Lance Ferris et al., 2019). Indeed, Bolino (1999) and Bolino, Turnley,
and Bloodgood (2002) state that there is a consensus that OCBs stem from
dispositions, motivation, and fairness perceptions.

Studies have also identified predictors of within-person OCB variance,
but again many of these are individual characteristics. Antecedents
include positive affect (Dalal, Lam, Weiss, Welch, \& Hulin, 2009;
Glomb, Bhave, Miner, \& Wall, 2011), job satisfaction (Ilies, Scott, \&
Judge, 2006), social comparisons and beliefs in a just world (Spence,
Ferris, Brown, \& Heller, 2011), core self evaluations and future
orientation (Wu \& Parker, 2012), engagement (Christian, Eisenkraft, \&
Kapadia, 2015), and perceptions of justice or supervisor support (Matta,
Sabey, Scott, Lin, \& Koopman, 2020; Schreurs, Hetty van Emmerik,
Günter, \& Germeys, 2012).

We offer an alternative, perhaps simpler, model to explain sustained,
superior levels of OCBs -- one that does not rely on individual
characteristics such as motives, attributions, personality, or fairness
perceptions. The mechanism, instead, uses (a) opportunities, or signals
that an act of assistance can be performed, and (b) chance accumulation,
or the notion of randomly assembling components to an existing stock as
an employee moves through time. To say that an employee randomly
accumulates opportunities is to mean that he or she is confronted with
requests, notifications, or prompts that signal to him or her that an
act of help can be performed, and each of these successive cases then
compiles into his or her existing pool. We show that whenever help
requests follow a random accumulation process, then superior, sustained
citizenship behaviors by one employee compared to others is not only a
possibility but in some cases it is the most likely outcome -- it is to
be expected. Even when two people have the same level of trust toward
others, empathy, or prosocial values, one may have continual, superior
helping due to the underlying, random accumulation. Moreover, this
result occurs even when the mechanism is identical for every employee.
In other words, we show that vastly different observed citizenship does
not depend on a unique causal diagram for every employee. The
fundamental process -- accumulation -- is the same, but the manifest
complexity leading some individuals to have greater citizenship than
others occurs due to the unique gradient one experiences across time.
Such an alternative explanation does not necessarily challenge existing
ideas, but it has the potential to change our understanding of what
generates sustained, superior behavior.

Apart from this first contribution, an alternative, parsimonious
explanation regarding sustained, superior citizenship, additional
contributions of this paper are as follows. First, we provide
information to managers that can help them avoid misattributing causes
of citizenship. If a manager were to take our literature at face value,
then she should assess individual characteristics to monitor, predict,
and manage helping behaviors. But such actions do not account for
differences in help requests and the extent to which these requests
follow a random process. Therefore, she cannot rule out chance when she
witnesses sustained, high levels of OCBs and will potentially
misattribute its cause to personality or motives. Any performance or
promotion recommendation that she then provides -- which are outcomes of
OCBs -- are given for the wrong reason. The employee behavior was not
due to disposition, but chance opportunities.

Second, we answer recent calls for a better understanding of dynamics in
the citizenship literature (DYNAMICS; dishop et al., DYNAMICS; cronin).
DYNAMICS; Dishop et al., argue that, although it is now common for
researchers to assess patterns in longitudinal data, many of the current
approaches miss several fundamental concepts of dynamics -- the notion
of accumulation being one. We examine this principle here by assessing
the extent to which help requests follow a random walk and therefore add
more knowledge about citizenship dynamics to our literature.

Third, we extend the OCB literature by examining the nature of help
requests. When researchers discuss employee citizenship in handbooks
(Podsakoff et al., 2018), theory (Bolino, Harvey, \& Bachrach, 2012;
Organ, 1988), or in empirical articles (Gabriel, Koopman, Rosen, \&
Johnson, 2018; Koopman, Lanaj, \& Scott, 2016), they focus almost
exclusively on help itself -- types, measures, predictors, outcomes, and
its similarity to other constructs. But help is often, sometimes by
definition, tied to a request or prompt. For instance, in their chapter
distinguishing OCBs from engagement, Newton and LePine (HANDBOOK; 2018)
suggest that citizenship is a response to an opportunity -- an act that
follows a prompt for extra work or a request for information. Similarly,
in their chapter distinguishing OCBs from proactive behavior, Li, Frese,
and Haider (HANDBOOK; 2018) state that, whereas proactive behavior
reflects an employee volunteering help without a prompt, OCBs are
actions that occur after a plea for assistance. Not all OCBs are
reactions to prompts (e.g., López-Domı'nguez, Enache, Sallan, \& Simo,
2013), but requests are part of the definition of at least one major
type of citizenship -- a type which some authors (Li et al., 2018) have
argued should take the forefront of OCB research. Currently, we have
many studies on helping but little on the nature of prompts. Our
understanding of citizenship, therefore, is incomplete in that we have
focused exclusively on one aspect of the definition of citizenship
(i.e., the act), and not the other (i.e., the prompt).

Fourth, we challenge an assumption about what creates long-run,
exceptional citizenship. To appreciate our stance, it is useful to
describe a study by Bolino et al. (2015). These authors examine
within-person variance in OCBs, depletion, and motives, and correlate
the constructs over time. They motivate their study by arguing that
prior research has assumed that people have stable motives and so
\enquote{good soldiers,} or employees that demonstrate supreme OCB
levels compared to their peers, will always be good. They argue that
this idea is unfounded and then demonstrate that motives do show
systematic within-person variance, and that they predict OCBs. What
these authors imply is that long-run behavior is unlikely when there is
systematic variance in the variables that are assumed to cause OCBs.
Said differently, when the causes are unstable (motives), the outcome
must be unstable (OCB). This idea, though, contradicts what we know
about stochastic processes, particularly the notion that no systematic
variance in the cause is required to produce what looks like long-run
stability in the outcome (Polson \& Scott, 2012). If the cause has no
systematic variance, it is still possible (and in some cases extremely
likely) that the response process does contain systematic variance in
the form of long streaks of exceptional citizenship. Our paper,
therefore, repositions how we think about long-run citizenship behavior.

Finally, this research generates new avenues; it points to unexplored
scientific and applied questions that could lead to a flurry of
additional work. These questions are unpacked at the end of the paper.

The goal of this paper is to describe an alternative, chance model of
long-run citizenship that incorporates opportunities and accumulation.
Below, we describe OCB background and theory, the notion of extra
milers/good soldiers, and then present our alternative explanation with
two studies. In study one, we propose and find evidence that help
requests follow a random accumulation process. Specifically, we draw
from probability theory and suggest that, in some cases, patterns of
help requests follow random walks. In study two, we use this initial
finding as a starting point -- that help requests can be modeled as
random walks -- and then apply simulations to determine how different
types of random walks lead to various forms of long-run behavior. Stated
simply, study two reveals the parameters and assumptions required for
random walks to produce long-run OCBs.

\hypertarget{organizational-citizenship-behaviors-ocbs}{%
\section{Organizational Citizenship Behaviors
(OCBs)}\label{organizational-citizenship-behaviors-ocbs}}

The idea that there are employee behaviors beyond what we typically
consider as job or task performance but that still promote individual
and collective success has been around for decades. Researchers from
psychology, management, education, human resources, organizational
behavior, and sociology have different terms for this behavior, and
different aspects that they emphasize, but in the organizational
literature this behavior has come to be known as organizational
citizenship. OCB is \enquote{individual behavior that is discretionary,
not directly or explicitly recognized by the formal reward system, and
that in the aggregate promotes the effective functioning of the
organization} {[}Organ (1988); p.~4{]}. It has been described as a
behavior that \enquote{lubricates} the social machinery of the
organization, thereby facilitating its effective functioning (Bolino et
al., 2002; Organ, Podsakoff, \& MacKenzie, 2005; Podsakoff \& MacKenzie,
1997). Related terms that are now less popular include organizational
spontaneity (George \& Brief, 1992), extra-role behavior (Van Dyne \&
LePine, 1998), and contextual performance (Motowidlo \& Van Scotter,
1994).

Citizenship has consequences for both individuals and collectives.
Employees demonstrating greater OCBs earn higher supervisor performance
evaluations (MacKenzie, Podsakoff, \& Fetter, 1991, 1993; Motowidlo \&
Van Scotter, 1994) and more promotion recommendations (Van Scotter,
Motowidlo, \& Cross, 2000). Meta-analytic results suggest that
individuals who consistently engage in OCB are less likely to express
intentions to leave, to voluntarily quit, and to be absent from work
(Podsakoff, Whiting, Podsakoff, \& Blume, 2009). For collectives,
greater levels of OCBs relate to higher performance quality, performance
quantity, and customer satisfaction (Ehrhart \& Naumann, 2004;
Podsakoff, MacKenzie, Paine, \& Bachrach, 2000), and some studies
suggest that organizations competing in changing environments are
especially dependent on good citizens because the goodwill and social
capital they foster are a source of competitive advantage (Bolino et
al., 2002; Leana \& van Buren, 1999; Nahapiet \& Ghoshal, 1998). There
are also studies documenting the negative consequences of OCBs, which
include reduced in-role performance, depletion and exhaustion, role
overload, and feelings of resentment among peers (HANDBOOK; cites). That
said, several researchers claim that OCBs should be thought of as a
positive act, which is highlighted in the following quotes:

\begin{quote}
\begin{quote}
\enquote{There is considerable support in the literature for the idea
that citizenship behavior at work is a positive thing} (Bolino et al.,
2015; p.~56)
\end{quote}
\end{quote}

\begin{quote}
\begin{quote}
\enquote{Theory and practice should acknowledge the sizable role good
citizens play\ldots{}because organizations rely on their continued
investments} (Methot et al., 2017; p.~11)
\end{quote}
\end{quote}

Researchers typically pursue one of three broad ways to classify OCBs.
Initially, OCB included two dimensions: altruism, or helping directed at
a person after an eliciting stimulus; and generlized compliance, or an
impersonal sense of conscientiousness (Smith, Organ, \& Near, 1983).
These two dimensions were later deconstructed into altriusm (responding
to opportunities to assist a coworker), courtesy (responding with
kindness), conscientioueness (on time, following rules, etc.), civic
virtue (concern for the organization), and sportsmanship (tolerate less
than ideal circumstances while maintaining a positive outlook)
(HANDBOOK). Other researchers classify OCBs either as affiliative or
challenging (HANDBOOK). Affiliative behaviors are acts such as helping
or responding with courtesy in which the actor supports existing company
processes. Challenging behaviors are acts such as voicing problems or
initiating change in which the actor adjusts his or her circumstances.
Finally, OCBs are also distinguished (e.g., Dalal, 2005) by whether they
benefit individuals (OCB-I; helping, assisting, encouraging) or the
organization (OCB-O; promoting the company to others).

In this paper, we refer to affialiative OCBs whenever we use the terms
citizenship, helping, acts of assistance, or OCB. This focus is
necessary and appropriate for the following reasons. First, Li and
HANDBOOKAUTHOR spend an entire chapter describing the differences
between affiliative (helping) and challenging (voicing) OCBs and argue
that helping should be thought of as the core manifestation of
citizenship because it (a) aligns with what most people mean when they
study cooperation in the broader sciences, (b) is based on different
evolutionary pressures than behaviors such as voicing concerns or
actively changing circumstances, and (c) leads to construct
contamination and unnecessary confusion if coupled with change-oriented
behaviors. Second, Van Dyne, Cummings, and McLean (1995 HANDBOOK BUT OWN
TEXT) suggest that \enquote{the conceptual definition and subsequent
operationalizations of OCBs should focus on citizenship behavior that is
affiliative\ldots{}and should not include challenging} (p.~274). Third,
helping is the core dimension discussed in the original paper exploring
the dimensionality of OCBs (Smith et al., 1983) and within Organ's
theoretical writing about the construct (HANDBOOK OWN TEXT ORGAN).
Finally, and perhaps most importantly, it aligns with the purpose of
this study, which is to explore the random nature of prompts for help.
For all of these reasons, this paper couches itself within the
affiliative space of the construct.

\hypertarget{sustained-long-run-citizenship}{%
\section{Sustained, Long-Run
Citizenship}\label{sustained-long-run-citizenship}}

Recently, researchers have shown an increasing interest in employees
that repeatedly exhibit greater OCBs compared to their peers. Li et al.
(2015), for instance, studied manufacturing teams in China and examined
what they referred to as \enquote{extra milers} -- employees who
frequently provide greater help relative to their colleagues.
Specifically, extra milers were defined as team members who exhibited
high frequency extra-role behaviors such as helping, and the researchers
operationalized it by collecting other-team-member-rated surveys of OCBs
and then identifying the team member with the maximum score.
Unfortunately, there was a discrepancy between how they defined extra
milers and how it was studied: they defined it by referring to
frequency, which implies sustained behavior over time consistent with
the theory that they used to support their arguments (behavioral
consistency theory), whereas the measures they employed captured
single-moment levels of OCBs. Nonetheless, the researchers were clearly
interested in the notion of repeated, exceptional OCBs. They found that
differences across teams in the number of helping behaviors provided by
the \enquote{extra miler} correlated with team backup and monitoring
behaviors.

A similar idea is described in a paper by Methot et al. (2017) that
explains how employees make sense of life events and its implications
for OCB. They state,

\begin{quote}
\begin{quote}
One topic of particular interest in the OCB literature is the concept of
\enquote{good citizens} -- employees who tend to engage in high levels
of OCB\ldots{} Research suggests that good citizens characteristically
perform OCB because of such factors as personality traits, including
agreeableness, prosocial orientation and values, and proactive
personality. p.~10.
\end{quote}
\end{quote}

So, good soldiers or extra milers refer to employees that
\enquote{characteristically} engage in OCB, or that exhibit greater
helping compared to their colleagues time and time again. Such a pattern
would manifest as long-run streaks of behavior, similar to a series of
consecutive heads if one were to flip a coin two hundred times.

What accounts for long-run citizenship? OCB antecedents were described
earlier in this paper and included individual characteristics such as
motives, affect, attitudes, fairness perceptions, and engagement.
Similarly, Methot et al., point to predictors of long-run citizneship in
the quote above: personality and prosocial values. We suggest an
alternative: chance opportunities. Just as a series of consecutive heads
could be a byproduct of chance events from flipping a coin, long-run
citizenship could be a byproduct of random opportunities. By
opportunity, we mean a prompt that signals to an employee that an act of
help can be performed, such as an email from a colleague requesting
assistance. By random, we mean that help requests follow a mathematical
form that incorporates chance. The overarching argument in this paper is
that employees may receive help requests in a pattern that mimics a
fundamental mathematical process, one that includes randomness, and so
in the sections below it is necessary to articulate each aspect of our
argument. First, we describe what we mean by help requests or
opportunities. Then, we provide one way to specify their mathematical
form.

\hypertarget{prompts-opportunities}{%
\section{Prompts \& Opportunities}\label{prompts-opportunities}}

A prompt/request/opportunity is a signal to an employee that an act of
help can be performed, and this idea was an important element in the
early OCB literature. In their cornerstone paper describing its
dimensions, Smith et al. (1983) state that helping occurs after a
stimuli, or a signal that \enquote{appears to be situational, that is,
someone has a problem, needs assistance, or requests a service}
(p.~661). Despite this initial emphasis, Ehrhart (HANDBOOK CHAPTER)
points out that there has been little follow-up research on the nature
of requests and how they inform what we know about OCBs. That said,
there is ample theory elsewhere that describes opportunities more
broadly as they reflect aspects of the situation or environment in which
an agent is conducting his or her behavior -- we draw from this
literature to guide our discussion.

Many researchers across several scientific disciplines have described
the nature of situations and environments. Within this broad area, two
ways to think about the environment are relevant for our purposes. The
first is as a platform, space, or zone which holds distributed
goal-relevant objects. This perspective is consistent with much of
Herbert Simon's writing that emphasized the importance of context for
understanding human behavior. Across a number of papers, theories, and
normative models (Simon, 1956, 1992) Simon argues that to understand the
complex behavior of an agent it is first necessary to understand how
goal-relevant objects are distributed around it. Applied to the current
paper, this notion embodies the idea that to understand OCBs it is
necessary to know how opportunities to assist are distributed about an
employee. To make his writing clear, Simon usually described how objects
were distributed in space, meaning that an agent was located in a matrix
and the distribution was over cells. Here, we extend that idea to a
distribution over time. Not only do employees receive help requests from
different colleagues, they also receive requests at different moments in
time, and requests happen repeatedly as an employee moves from moment to
moment. This distribution over time would reflect the average number of
requests that the employee would expect to receive at any moment,
alongisde the expected variability in requests.

The second perspective on the environment is as a shock or disturbance
that makes opportunities come and go. Random stimuli occur and these
factors impinge upon actors, allowing some behaviors and constraining
others. This idea is consistent with the notion of shocks in the
unfolding model of employee turnover in which discrete events thwart
some opportunities and create others (Lee \& Mitchell, 1994), to events
in affective events theory in which random stimuli cause changes in
employee emotion and behavior (Beal, Weiss, Barros, \& MacDermid, 2005),
and to the environment in Dishop's goal sampling theory (Dishop, 2019)
in which actors are only able to approach goals made available by the
situation at any moment in time. Blumberg and Pringle (1982) define
opportunities as \enquote{the particular configuration of the field of
forces surrounding a person and his or her task that enables or
constrains that person's task performance and that are beyond the
person's direct control} (p.~565), and Stewart and Nandkeolyar (2007)
demonstrated that even skilled and motivated workers cannot engage in
performance facilitating behavior when their actions are constrained by
the environment.

Across all of these perspectives, the core idea is that there are
opportunities scattered about the environment that come and go. The
particular form of opportunity that we examine in this study is a help
request: a prompt or signal or notification to an employee that an act
of assistance can be performed. Consider a few examples: A Professor
receives an email from a colleague asking if she can substitute for an
undergraduate course; A manager announces that volunteers are needed for
an upcoming assignment; A blogger tells his writing collaborator that
she is welcome to review and edit his post if she pleases; A
statistician witnesses a question posted on a forumn about a statistical
model relevant to her expertise; A software engineer receives a pull
request; An academic receives a note from a graduate student asking for
a friendly review of his paper. Moreover, any single agent may
experience repeated prompts over the course of a week. On Monday, a
Professor may receive an email asking for assistance teaching a class.
On Tuesday, she receives two more emails about optional meetings in her
department (attending optional meetings is one commonly studied
manifestation of OCB). On Wednesday, a former graduate student, who is
now a faculty member at a different school, asks for a letter of
recommendation. On some days, the Professor has a large stock of help
requests, whereas on others she has few, if any. The crux of this paper
is that we expect these helping prompts to follow a specific
mathematical form, which we specify below.

\hypertarget{accumulating-requests-as-a-random-walk}{%
\section{Accumulating Requests As a Random
Walk}\label{accumulating-requests-as-a-random-walk}}

To explain patterns in help requests over time, we draw from probability
theory. For some employees, the pattern by which they receive help
requests may mimic a fundamental mathematical process. To see how,
consider the following heuristic. First, the state we are tracking is
the number of help requests than an employee receives over time, with
greater values indicating more notifications. Second, this state can be
viewed as a dynamic stock, meaning that the employee has a pool or store
of help requests -- three, for example -- and this number is
self-similar over time such that it carries over from day to day. If the
employee receives two help requests today, this number is added to the
store of help requests that she had yesterday, creating a total that
moves forward into tomorrow. Similarly, when help requests are removed
from the pool -- which could occur, for instance, after she or someone
else provides help and the request is resolved or when a deadline passes
and help is no longer required -- then it decreases by whatever amount
was withdrawn. But removing a request does not drive the pool to zero.
Instead, whatever amount was removed is subtracted from the total in
such a way that the pool has inertia/memory -- the amount changes from
where it was at the immediately prior time point, it does not
arbitrarily swing to zero. This pattern, one in which an employee
handles a dynamic stock of help requests such that prompts are added or
removed while the stock retains inertia, mimics a common and simple
stochastic process: a random walk.

A random walk is a basic concept from probability theory. Models of
random walks have been used in many scientific disciplines ranging from
physics, biology, and chemistry (Kenkre, Montroll, \& Shlesinger, 1973;
Kot, Medlock, Reluga, \& Walton, 2004; Randić, 1980) to economics,
sociology, and psychology (Alvarez, Atkeson, \& Kehoe, 2007; Johnson,
2014; Shang, 2018), helping to understand diverse phenomenon such as
memory search (Stamovlasis \& Tsaparlis, 2003), particle motion (Bramson
\& Lebowitz, 1991), network and market behavior (Fama, 1995; Newman,
2005), and animal foraging (Sims et al., 2014).

A random walk is defined as.

\begin{equation}
y_{t} = y_{t-1} + B + e_{t}
\end{equation}

where \(y_{t}\) is the current value of \(y\), \(y_{t-1}\) is the value
of \(y\) at \(t - 1\), \(B\) is a constant known as drift, and \(e_{t}\)
is a series with a mean zero and constant variance \(\sigma^2_{e}\).
This first equation reveals that random walks contain inertia or
self-similarity, which is consistent with our heuristic of helping
prompts above. Although drift and error are involved, the core aspect of
a random walk as represented in equation 1 is that the value of \(y\) at
a given time point is a function of its value at the immediately prior
time point.

Another key aspect of random walks is that they incorporate
accumulation, which is more readily apparanet in an alternative but
equivalent form:

\begin{equation}
y_{t} = y_{0} + Bt + \sum_{i = 1}^{t} \varepsilon_i
\end{equation}

where \(y_{0}\) is the initial value of \(y\), \(Bt\) is a deterministic
trend component, and the last term represents an accumulation of error.
This second equation reveals that random walks capture the notion of
accumulating, or adding values to a store/pool over time, which was the
second component to our heuristic of help requests.

In the same way that logic can be excavated from a verbal theory to gain
traction about some phenomenon, the notion of a random walk can be drawn
from probability theory to better understand the nature of help
requests. Specifically, we suggest that help requests follow a random
walk, such that they demonstrate self-similarity and have the
characteristic of accumulating over time.

\begin{quote}
\begin{quote}
Hypothesis 1: Help requests follow a random walk.
\end{quote}
\end{quote}

In study one, we examine a number of data sources to evaluate whether we
can find evidence that help requests follow this stochastic process.

\hypertarget{study-1}{%
\section{Study 1}\label{study-1}}

Archival data was used to assess our hypothesis. We scraped data from
several different sources on the Internet, each capturing the idea of a
help request in a slightly different way. Testing for random walks
requires time-series data with many time points (\(t\)), therefore we
searched for platforms that contained data with large \(t\) and that
could be used to capture notifications for help.

\hypertarget{data-sources}{%
\subsection{Data Sources}\label{data-sources}}

\begin{enumerate}
\def\labelenumi{(\arabic{enumi})}
\tightlist
\item
  Issues on GitHub Repositories - Non-Academic
\end{enumerate}

The first set of data was collected from GitHub repositories created by
software developers. GitHub is an open source website that allows users
to store, manage, share, and collaborate on projects (repositories) and,
although most use it for code, it can also be used for other types of
documents such as Word files. The website contains a variety of features
that facilitate transparency, collaboration, and networking on projects,
such as version control, the ability to comment on and request edits to
other user's projects, and personal pages that exhibit a given user's
track-record of work. The data that we collected are known as repository
\enquote{issues.} When a focal individual posts a repository/project,
other users can then download and use the code that he created. If other
users want to ask questions about the code, request features, or report
bugs, they can then create an issue on the focal individual's post. The
focal individual is then notified that an issue has been placed.

The data we collected were issues posted to single repositories, and we
collected data on four different software developers. That is, a single
software developer had a respository that he or she maintained, and over
time his or her repository collected issues. All of the issues, from
when the project first began until the most recent comment, were
collected and time-stamped. This process was then repeated for another
three software developers working in different industries on unrelated
projects.

One of the repositories was source code for a functional computer
language built to create web applications. Another was a compiler to
convert declarative components into JavaScript. The third was an
application which corrects console commands. The fourth was a facial
recognition application programming interface. Three of the four
software developers work full time for a given company, whereas the
fourth is an external consultant.

For each data set, help opportunities were operationalized as issues.
Data were collected on (a) the date that the issue was posted and (b)
when it was resolved, if ever.

\begin{enumerate}
\def\labelenumi{(\arabic{enumi})}
\setcounter{enumi}{1}
\tightlist
\item
  Issues on GitHub Repositories - Academic
\end{enumerate}

The second set of data was also based on GitHub repositories, but this
time we used repositories posted by academics. University faculty often
use GitHub as a version control system when writing documents, as a
platform to share, monitor, and adjust any applications or tools that
they develop, and as a resource for downloading data science tools. We
focused on the individual repositories of four academics, each a faculty
member at a different university.

One of the repositories was an R package for structural equations
modeling. Another was the source code and package for a popular Bayesian
analysis textbook. The third was an R package for multivariate analysis
of genetic markers, and the fourth was a package for population
genetics. As before, help opportunities were operationalized as issues
and we collected (a) the time the issue was placed and (b) when, if
ever, it was resolved.

\hypertarget{potential-data-sources}{%
\subsubsection{Potential Data Sources}\label{potential-data-sources}}

\begin{enumerate}
\def\labelenumi{(\arabic{enumi})}
\setcounter{enumi}{2}
\tightlist
\item
  first author own emails
\end{enumerate}

The third set of data was a series of emails received by the first
author. From October, 2019 to August, 2020, the first author saved any
emails from colleagues that seemed relevant to the notion of helping
opportunities. This process was not systematic on the front end: the
author stored emails based on his own discretion, storing only those
emails that appeared relevant as they were received. We tried to make
the process more systematic on the back end: after collecting all of the
emails and removing any identifying information, 300 undergraduate
students underwent a sorting procedure in which they classified the
emails either as helping opportunities or as irrelevant. We describe
this process in more detail below.

Three hundred undergraduates at a large Midwestern university were
recruited to take part in a classification study, which participants
completed online. After giving consent, the participants were provided
with a definition of helping opportunities and several example items
used in prior empirical research. They were then presented with the
content of a single email, asked to read it, and then were told to
determine if the content was consistent with a helping opportunity or
not. Participants rated each email with a bipolar scale including
\enquote{yes} or \enquote{no.} Agreement indices were collected. In this
data set, help opportunities were operationalized as emails that raters
agreed represented requests for citizenship.

\begin{enumerate}
\def\labelenumi{(\arabic{enumi})}
\setcounter{enumi}{3}
\tightlist
\item
  student pools
\end{enumerate}

Our fourth angle on help opportunities came from graduate student pools.
We tracked the number of graduate students per year from the years 1999
to 2019 at three different graduate programs. One was a Political
Science program located in the Northeast, another was an Organizational
Psychology department located in the Midwest, and the third program was
in Accounting and located in the Southwest. In this data set, a help
opportunity was operationalized as an active graduate student -- someone
who could be mentored by a faculty -- and we collected data on the
number of active graduate students per year for each department.

\begin{enumerate}
\def\labelenumi{(\arabic{enumi})}
\setcounter{enumi}{4}
\tightlist
\item
  forumn questions
\end{enumerate}

Finally, we also collected data from an online forumn.
\enquote{Psychological Dynamics} is a Facebook group which provides
users with a platform to share and discuss news, publications, tools,
and other aspects related to psychological research. The community draws
researchers from all over the world, and posts are created every day. In
this data set, help opportunities were operationlized as a post, and
posts were collected daily from September, 2018 to September, 2019.

{[}table with each data type{]}

A summary of the data sources is presented in Table 1. We collected data
across diverse platforms for several reasons. First, we wanted to ensure
that our results were not unique to a given domain. Just as Aguinis and
colleagues demonstrated performance power curves in different settings,
our goal was to demonstrate random walks across various platforms.
Second, we collected data from several sources because each has its own
unique limitations and strengths. Our hope was that we could learn
something about help requests in general by taking a broad view across
all of the data, even though each has its own unique error. The set as a
whole can tell us something about help requests, even if each has a
slight weakness.

\hypertarget{analysis}{%
\subsection{Analysis}\label{analysis}}

All data are structured as time-series such that a single unit is
represented over successive time points. In total, there are 13 data
sets: 8 from the GitHub repositories, 1 from the first author's emails,
an additional 3 from the PhD student pools, and 1 from the public
forumn. Each of these time-series represents the stock of help
opportunities over time, such that greater values indicate more helping
opportunities and lower values indicate fewer helping opportunities. For
each data set, hypothesis one is evaluated by assessing whether the
series contains a unit root. We use two unit root tests to evaluate our
hypothesis. The first, the augmented Dickey-Fuller (ADF; Dickey \&
Fuller, 1979) test, is the most widely used statistic to evaluate the
presence of random walks in time-series data. The null hypothesis of
this test is that the data are generated from a random walk, so when the
ADF test cannot reject its null our hypothesis is retained. There are
also unit root tests in which the null hypothesis is instead the absence
of a unit root, and the most well-known test of this second type is the
Kwiatkowski, Phillips, Schmidt, Shin, and others (1992) statistic
(KPSS). Both tests were administered to evaluate our hypothesis. Stated
simply, if the ADF test cannot reject its null while the KPSS test can,
then the data provide evidence in two ways that the series follows a
random walk.

\hypertarget{results}{%
\subsection{Results}\label{results}}

{[}Fill after data collection{]}.

\hypertarget{study-1-discussion}{%
\subsection{Study 1 Discussion}\label{study-1-discussion}}

Study one demonstrated that, at least in some cases, help opportunities
can be modeled as random walks. Time-series data were collected from
multiple sources, and each series represented an accumulating pattern of
help opportunities over time. For 12 out of the 13 data sources, both
unit root tests provided evidence that the series was consistent with a
random walk. In the last data set, which consisted of \(X\), only the
ADF test returned evidence that a random walk was present. Identifying
random patterns in help requests was the first step toward our chance
model of long-run citizenship. We take this evidence -- that help
opportunities follow a random walk -- as a starting point for our next
study.

\hypertarget{study-2}{%
\subsection{Study 2}\label{study-2}}

Our second study reveals the ways in which random walks may produce
different forms of long-run citizenship. Its purpose is to document
patterns of long-run citizenship that emerge from different types of
random walks. Given that we identified random walks in study one, the
next step is to assess how varying the parameters of random walks, as
well as our assumptions about the connection between opportunities and
acts of help, changes the extent to which they produce extra milers. We
pursue this study by using simulations, which allow us to witness the
effects of varying crucial paramters in systematic ways. First, though,
it is necessary to articulate again the idea of extra milers and
long-run citizenship.

There are two phrases in the litarture that researchers have used to
describe long-run citizenship: good soldiers and extra milers. Methot et
al. (2017) state that good soldiers are people who characteristically
engage in higher levels of OCB relative to their colleagues. They are
people who provide more help, relative to others, in
\enquote{characteristic} ways. Similarly, Li et al. (2015)
operationalized extra milers as employees who provided the most (as
rated by team members) OCBs at a given time point, even though their
theoretical definition of extra milers were those that had this maximum
score across repeated time points. How would these ideas manifest? What
is implied in how the researchers describe, study, and label this
phenomoneon -- which we refer to here as long-run citizenship -- is that
some employees perform more OCBs than their peers and this effect has
some form of consistency. At time \(t\), the individual performs more
OCBs than her colleagues, she does so again at time \(t + 1\), again at
\(t + 2\), and this pattern continues until \(t + n\), \(n\) being any
future time point in which she is outdone by a colleague. The value of
\(n\) that determines whether a person is labeled as an extra miler or
not remains unspecified, as does the number of consecutive
\enquote{wins} required. Said differently, it is unclear for how long
someone must sit as the top citizen to be considered an extra miler, and
it is also unclear whether the streaks must be consecutive or if someone
who is frequently a top citizen but never the top citizen for more than
two time points in a row merits the label. Our interest is not in
providing a definition or argument about what truly does and does not
count for employees to be labeled as extra milers by researchers or
their managers. Our interest, instead, is on what kinds of streaks
emerge given different random walk parameters and different assumptions
about the relationship between opportunities and OCBs. What types of
streaks, or consecutive \enquote{wins} by one colleague compared to
another with respect to their helping behavior, do we witness under
different random walks? Our research is the start to creating benchmarks
that can be used in later research to determine what is really required
to label something as exceptional.

Again, the purpose of study two is to assess patterns in long-run
citizenship, or the extent to which one individual provides greater help
compared to others across consecutive time points, based on different
parameters applied to help requests. We use simulations for this study,
and the computer models are structured as follows.

\hypertarget{simulation-heuristic}{%
\subsection{Simulation Heuristic}\label{simulation-heuristic}}

The simulation was designed to build off prior research examining chance
models and accumulating processes in areas such as firm performance
(Denrell, 2004; Polson \& Scott, 2012). Imagine two employees, each
collecting help requests according to a random walk. From \(t\) to
\(t + 1\), each employee retains his or her stock of help requests but
the pool increases or decreases by an amount drawn from a stochastic
term, meaning that the value by which it increases or decreases is
random at each moment. This structure exactly mimics the random walks
identified in study one. At any given time point, help requests lead to
helping such that the employee with the greatest number of opportunities
provides the most help. Mathematically, if
\((x_i, x_{i+1}, \ldots, x_n)\) represents the set of employees whose
help requests we are tracking over time, with \(x_i\) being the focal
employee, then \(x_i\) provides the most help at time \(t\) when
\(x_i > x_n\). We refer to the employee that provides the most help at a
given time the \enquote{moment citizen,} which naturally embodies the
idea of a single time point. The pattern that we monitor is the number
of consecutive times employee \(x_i\) is the moment citizen, which ties
back to the notions of good soldier and extra miler. Said differently,
the ideas represented in how researchers have described good soldiers
and extra milers manifests whenever employee \(x_i\) has long-run
streaks of being the moment citizen, whereas if the moment citizen
changes from time point to time point, such that no long-run streaks
emerge, then the random walks produced no evidence of these labels.
Ultimately, we are asking the broad question, \emph{What types of
long-run streaks do we witness when we vary the parameters on random
walks?} We conducted synthetic experiments, or experiments within a
computer program in which we wiggle key parameters and witness the
output, to tackle this question. Moreover, the parameters that we
manipulate stem from three research questions.

\begin{quote}
\begin{quote}
Research Question 1: What are the patterns in long-run citizenship as
the drift parameter on helping opportunities changes from 0 to 1?
\end{quote}
\end{quote}

Research question 1 was designed to address how trending help requests
change the results. A trend or drift term is an essential property of a
random walk, although not all random walks have drift. Drift, or trend,
refers to whether the random walk moves systematically in the positive
or negative direction over time, despite moving stochastically at each
time step. Random walks without drift, conversely, move randomly from
moment to moment but do not show positive or negative trend, unless cut
short due to sampling limitations. Plotting the random walks from study
one revealed that both types occurred, so it is necessary to evaluate
how this characteristic informs our results. Moreover, an ever-growing
amount of evidence (Braun, Kuljanin, \& DeShon, 2013; Kuljanin, Braun,
\& DeShon, 2011) suggests that researchers need to account for the
implications of stochastic trends in their content areas whenever
effects are explored over time, so it is a crucial aspect to incorporate
here.

\begin{quote}
\begin{quote}
Research Question 2: What are the patterns in long-run citizenship as
the autoregressive parameter on helping opportunities changes from 0 to
1?
\end{quote}
\end{quote}

One fundamental characteristic of random walks is that they have strong
autoregressive effects. As this effect goes to zero, the trajectory
approaches a white noise process, which is another fundamental
stochastic trajectory. The difference is that white noise processes only
move according to the error term -- they contain no self-similarity from
moment to moment. We examine this feature because (a) it captures the
essence of what it means for opportunities to follow a random walk and
(b) is consistent with growing calls to examine the implications of
different strengths of self-similarity among dynamic trajectories
(DISHOP HANDBOOK).

\begin{quote}
\begin{quote}
Research Question 3: What are the patterns in long-run citizenship as
the number of employees in the simulation increase from 2 to 1000?
\end{quote}
\end{quote}

Research question 3 was designed to assess how the size of the
collective influences patterns in long-run citizenship. Organizational
science has been and continues to be a science focused on individual
differences and collectives. Nearly all studies in the organizational
literature are multiple unit, meaning that they examine their effects
across multiple people, teams, departments, or companies. This effect
was therefore important to examine given the collective nature of our
field.

\hypertarget{analysis-results}{%
\subsection{Analysis \& Results}\label{analysis-results}}

{[}Fill after data collection{]}. {[}Some expected results shown in
tables 1-3{]}.

table 1 - data source, help opportunity operationalization, sampling
frequency

Or I could just do code. Code for base simulation - how will the code
change for rq1? - how will the code change for rq2? - how will the code
change for rq3?

figure 1 - expected result from rq1 simulation

figure 2 - expected result from rq2 simulation

figure 3 - expected result from rq3

\newpage

\hypertarget{references}{%
\section{References}\label{references}}

\setlength{\parindent}{-0.5in}
\setlength{\leftskip}{0.5in}

\hypertarget{refs}{}
\leavevmode\hypertarget{ref-alvarez2007if}{}%
Alvarez, F., Atkeson, A., \& Kehoe, P. J. (2007). If exchange rates are
random walks, then almost everything we say about monetary policy is
wrong. \emph{American Economic Review}, \emph{97}(2), 339--345.

\leavevmode\hypertarget{ref-bateman_job_1983}{}%
Bateman, T. S., \& Organ, D. W. (1983). Job satisfaction and the good
soldier: The relationship between affect and employee ``citizenship''.
\emph{Academy of Management Journal}, \emph{26}(4), 587--595.

\leavevmode\hypertarget{ref-beal_episodic_2005}{}%
Beal, D. J., Weiss, H. M., Barros, E., \& MacDermid, S. M. (2005). An
episodic process model of affective influences on performance.
\emph{Journal of Applied Psychology}, \emph{90}(6), 1054.

\leavevmode\hypertarget{ref-blumberg1982missing}{}%
Blumberg, M., \& Pringle, C. D. (1982). The missing opportunity in
organizational research: Some implications for a theory of work
performance. \emph{Academy of Management Review}, \emph{7}(4), 560--569.

\leavevmode\hypertarget{ref-bolino_citizenship_1999}{}%
Bolino, M. C. (1999). Citizenship and impression management: Good
soldiers or good actors? \emph{Academy of Management Review},
\emph{24}(1), 82--98.

\leavevmode\hypertarget{ref-bolino_self-regulation_2012}{}%
Bolino, M. C., Harvey, J., \& Bachrach, D. G. (2012). A self-regulation
approach to understanding citizenship behavior in organizations.
\emph{Organizational Behavior and Human Decision Processes},
\emph{119}(1), 126--139.
doi:\href{https://doi.org/10.1016/j.obhdp.2012.05.006}{10.1016/j.obhdp.2012.05.006}

\leavevmode\hypertarget{ref-bolino_well_2015}{}%
Bolino, M. C., Hsiung, H.-H., Harvey, J., \& LePine, J. A. (2015).
``Well, I'm tired of tryin'!'' Organizational citizenship behavior and
citizenship fatigue. \emph{Journal of Applied Psychology},
\emph{100}(1), 56.

\leavevmode\hypertarget{ref-bolino_citizenship_2002}{}%
Bolino, M. C., Turnley, W. H., \& Bloodgood, J. M. (2002). Citizenship
behavior and the creation of social capital in organizations.
\emph{Academy of Management Review}, \emph{27}(4), 505--522.

\leavevmode\hypertarget{ref-bramson1991asymptotic}{}%
Bramson, M., \& Lebowitz, J. L. (1991). Asymptotic behavior of densities
for two-particle annihilating random walks. \emph{Journal of Statistical
Physics}, \emph{62}(1-2), 297--372.

\leavevmode\hypertarget{ref-braun_spurious_2013}{}%
Braun, M. T., Kuljanin, G., \& DeShon, R. P. (2013). Spurious Results in
the Analysis of Longitudinal Data in Organizational Research.
\emph{Organizational Research Methods}, \emph{16}(2), 302--330.
doi:\href{https://doi.org/10.1177/1094428112469668}{10.1177/1094428112469668}

\leavevmode\hypertarget{ref-christian2015dynamic}{}%
Christian, M. S., Eisenkraft, N., \& Kapadia, C. (2015). Dynamic
associations among somatic complaints, human energy, and discretionary
behaviors: Experiences with pain fluctuations at work.
\emph{Administrative Science Quarterly}, \emph{60}(1), 66--102.

\leavevmode\hypertarget{ref-dalal2005meta}{}%
Dalal, R. S. (2005). A meta-analysis of the relationship between
organizational citizenship behavior and counterproductive work behavior.
\emph{Journal of Applied Psychology}, \emph{90}(6), 1241.

\leavevmode\hypertarget{ref-dalal_within-person_2009}{}%
Dalal, R. S., Lam, H., Weiss, H. M., Welch, E. R., \& Hulin, C. L.
(2009). A within-person approach to work behavior and performance:
Concurrent and lagged citizenship-counterproductivity associations, and
dynamic relationships with affect and overall job performance.
\emph{Academy of Management Journal}, \emph{52}(5), 1051--1066.

\leavevmode\hypertarget{ref-denrell2004random}{}%
Denrell, J. (2004). Random walks and sustained competitive advantage.
\emph{Management Science}, \emph{50}(7), 922--934.

\leavevmode\hypertarget{ref-deshon_multivariate_2012}{}%
DeShon, R. P. (2012). Multivariate dynamics in organizational science.
\emph{The Oxford Handbook of Organizational Psychology}, \emph{1},
117--142.

\leavevmode\hypertarget{ref-dickey_distribution_1979}{}%
Dickey, D. A., \& Fuller, W. A. (1979). Distribution of the estimators
for autoregressive time series with a unit root. \emph{Journal of the
American Statistical Association}, \emph{74}(366a), 427--431.

\leavevmode\hypertarget{ref-dishop_simple_2019}{}%
Dishop, C. R. (2019). A simple, dynamic extension of temporal motivation
theory. \emph{The Journal of Mathematical Sociology}, 1--16.

\leavevmode\hypertarget{ref-ehrhart2004organizational}{}%
Ehrhart, M. G., \& Naumann, S. E. (2004). Organizational citizenship
behavior in work groups: A group norms approach. \emph{Journal of
Applied Psychology}, \emph{89}(6), 960.

\leavevmode\hypertarget{ref-fama1995random}{}%
Fama, E. F. (1995). Random walks in stock market prices. \emph{Financial
Analysts Journal}, \emph{51}(1), 75--80.

\leavevmode\hypertarget{ref-gabriel_helping_2018}{}%
Gabriel, A. S., Koopman, J., Rosen, C. C., \& Johnson, R. E. (2018).
Helping others or helping oneself? An episodic examination of the
behavioral consequences of helping at work. \emph{Personnel Psychology},
\emph{71}(1), 85--107.

\leavevmode\hypertarget{ref-george1992feeling}{}%
George, J. M., \& Brief, A. P. (1992). Feeling good-doing good: A
conceptual analysis of the mood at work-organizational spontaneity
relationship. \emph{Psychological Bulletin}, \emph{112}(2), 310.

\leavevmode\hypertarget{ref-glomb_doing_2011}{}%
Glomb, T. M., Bhave, D. P., Miner, A. G., \& Wall, M. (2011). Doing
Good, Feeling Good: Examining the Role of Organizational Citizenship
Behaviors in Changing Mood. \emph{Personnel Psychology}, \emph{64}(1),
191--223.
doi:\href{https://doi.org/10.1111/j.1744-6570.2010.01206.x}{10.1111/j.1744-6570.2010.01206.x}

\leavevmode\hypertarget{ref-grant_does_2008}{}%
Grant, A. M. (2008). Does intrinsic motivation fuel the prosocial fire?
Motivational synergy in predicting persistence, performance, and
productivity. \emph{Journal of Applied Psychology}, \emph{93}(1), 48.

\leavevmode\hypertarget{ref-grant_good_2009}{}%
Grant, A. M., \& Mayer, D. M. (2009). Good soldiers and good actors:
Prosocial and impression management motives as interactive predictors of
affiliative citizenship behaviors. \emph{Journal of Applied Psychology},
\emph{94}(4), 900--912.
doi:\href{https://doi.org/10.1037/a0013770}{10.1037/a0013770}

\leavevmode\hypertarget{ref-hafenbrack_helping_2019}{}%
Hafenbrack, A. C., Cameron, L. D., Spreitzer, G. M., Zhang, C., Noval,
L. J., \& Shaffakat, S. (2019). Helping People by Being in the Present:
Mindfulness Increases Prosocial Behavior. \emph{Organizational Behavior
and Human Decision Processes}, S0749597817308956.
doi:\href{https://doi.org/10.1016/j.obhdp.2019.08.005}{10.1016/j.obhdp.2019.08.005}

\leavevmode\hypertarget{ref-ilies_interactive_2006}{}%
Ilies, R., Scott, B. A., \& Judge, T. A. (2006). The interactive effects
of personal traits and experienced states on intraindividual patterns of
citizenship behavior. \emph{Academy of Management Journal},
\emph{49}(3), 561--575.

\leavevmode\hypertarget{ref-johnson2014offenders}{}%
Johnson, S. D. (2014). How do offenders choose where to offend?
Perspectives from animal foraging. \emph{Legal and Criminological
Psychology}, \emph{19}(2), 193--210.

\leavevmode\hypertarget{ref-kenkre1973generalized}{}%
Kenkre, V., Montroll, E., \& Shlesinger, M. (1973). Generalized master
equations for continuous-time random walks. \emph{Journal of Statistical
Physics}, \emph{9}(1), 45--50.

\leavevmode\hypertarget{ref-koopman_integrating_2016}{}%
Koopman, J., Lanaj, K., \& Scott, B. A. (2016). Integrating the Bright
and Dark Sides of OCB: A Daily Investigation of the Benefits and Costs
of Helping Others. \emph{Academy of Management Journal}, \emph{59}(2),
414--435.
doi:\href{https://doi.org/10.5465/amj.2014.0262}{10.5465/amj.2014.0262}

\leavevmode\hypertarget{ref-kot2004stochasticity}{}%
Kot, M., Medlock, J., Reluga, T., \& Walton, D. B. (2004).
Stochasticity, invasions, and branching random walks. \emph{Theoretical
Population Biology}, \emph{66}(3), 175--184.

\leavevmode\hypertarget{ref-kuljanin_cautionary_2011}{}%
Kuljanin, G., Braun, M. T., \& DeShon, R. P. (2011). A cautionary note
on modeling growth trends in longitudinal data. \emph{Psychological
Methods}, \emph{16}(3), 249--264.
doi:\href{https://doi.org/http://dx.doi.org.proxy2.cl.msu.edu/10.1037/a0023348}{http://dx.doi.org.proxy2.cl.msu.edu/10.1037/a0023348}

\leavevmode\hypertarget{ref-kwiatkowski1992testing}{}%
Kwiatkowski, D., Phillips, P. C., Schmidt, P., Shin, Y., \& others.
(1992). Testing the null hypothesis of stationarity against the
alternative of a unit root. \emph{Journal of Econometrics},
\emph{54}(1-3), 159--178.

\leavevmode\hypertarget{ref-lance_ferris_being_2019}{}%
Lance Ferris, D., Fatimah, S., Yan, M., Liang, L. H., Lian, H., \&
Brown, D. J. (2019). Being sensitive to positives has its negatives: An
approach/avoidance perspective on reactivity to ostracism.
\emph{Organizational Behavior and Human Decision Processes}, \emph{152},
138--149.
doi:\href{https://doi.org/10.1016/j.obhdp.2019.05.001}{10.1016/j.obhdp.2019.05.001}

\leavevmode\hypertarget{ref-leana_organizational_1999}{}%
Leana, C. R., \& van Buren, H. J. (1999). Organizational Social Capital
and Employment Practices. \emph{The Academy of Management Review},
\emph{24}(3), 538.
doi:\href{https://doi.org/10.2307/259141}{10.2307/259141}

\leavevmode\hypertarget{ref-lee_alternative_1994}{}%
Lee, T. W., \& Mitchell, T. R. (1994). An Alternative Approach: The
Unfolding Model of Voluntary Employee Turnover. \emph{The Academy of
Management Review}, \emph{19}(1), 51--89.
doi:\href{https://doi.org/10.2307/258835}{10.2307/258835}

\leavevmode\hypertarget{ref-li_achieving_2015}{}%
Li, N., Zhao, H. H., Walter, S. L., Zhang, X.-a., \& Yu, J. (2015).
Achieving more with less: Extra milers' behavioral influences in teams.
\emph{Journal of Applied Psychology}, \emph{100}(4), 1025--1039.
doi:\href{https://doi.org/http://dx.doi.org.proxy1.cl.msu.edu/10.1037/apl0000010}{http://dx.doi.org.proxy1.cl.msu.edu/10.1037/apl0000010}

\leavevmode\hypertarget{ref-lopez2013transformational}{}%
López-Domı'nguez, M., Enache, M., Sallan, J. M., \& Simo, P. (2013).
Transformational leadership as an antecedent of change-oriented
organizational citizenship behavior. \emph{Journal of Business
Research}, \emph{66}(10), 2147--2152.

\leavevmode\hypertarget{ref-mackenzie1991organizational}{}%
MacKenzie, S. B., Podsakoff, P. M., \& Fetter, R. (1991). Organizational
citizenship behavior and objective productivity as determinants of
managerial evaluations of salespersons' performance.
\emph{Organizational Behavior and Human Decision Processes},
\emph{50}(1), 123--150.

\leavevmode\hypertarget{ref-mackenzie1993impact}{}%
MacKenzie, S. B., Podsakoff, P. M., \& Fetter, R. (1993). The impact of
organizational citizenship behavior on evaluations of salesperson
performance. \emph{Journal of Marketing}, \emph{57}(1), 70--80.

\leavevmode\hypertarget{ref-matta_not_2020}{}%
Matta, F. K., Sabey, T. B., Scott, B. A., Lin, S.-H. (., \& Koopman, J.
(2020). Not all fairness is created equal: A study of employee
attributions of supervisor justice motives. \emph{Journal of Applied
Psychology}, \emph{105}(3), 274--293.
doi:\href{https://doi.org/http://dx.doi.org.proxy2.cl.msu.edu/10.1037/apl0000440}{http://dx.doi.org.proxy2.cl.msu.edu/10.1037/apl0000440}

\leavevmode\hypertarget{ref-meglino_considering_2004}{}%
Meglino, B. M., \& Korsgaard, A. (2004). Considering rational
self-interest as a disposition: Organizational implications of other
orientation. \emph{Journal of Applied Psychology}, \emph{89}(6), 946.

\leavevmode\hypertarget{ref-methot_good_2017}{}%
Methot, J. R., Lepak, D., Shipp, A. J., \& Boswell, W. R. (2017). Good
Citizen Interrupted: Calibrating a Temporal Theory of Citizenship
Behavior. \emph{Academy of Management Review}, \emph{42}(1), 10--31.
doi:\href{https://doi.org/10.5465/amr.2014.0415}{10.5465/amr.2014.0415}

\leavevmode\hypertarget{ref-mitchell_can_2019}{}%
Mitchell, M. S., Greenbaum, R. L., Vogel, R. M., Mawritz, M. B., \&
Keating, D. J. (2019). Can You Handle the Pressure? The Effect of
Performance Pressure on Stress Appraisals, Self-regulation, and
Behavior. \emph{Academy of Management Journal}, \emph{62}(2), 531--552.
doi:\href{https://doi.org/10.5465/amj.2016.0646}{10.5465/amj.2016.0646}

\leavevmode\hypertarget{ref-motowidlo1994evidence}{}%
Motowidlo, S. J., \& Van Scotter, J. R. (1994). Evidence that task
performance should be distinguished from contextual performance.
\emph{Journal of Applied Psychology}, \emph{79}(4), 475.

\leavevmode\hypertarget{ref-nahapiet_social_1998}{}%
Nahapiet, J., \& Ghoshal, S. (1998). Social capital, intellectual
capital, and the organizational advantage. \emph{Academy of Management
Review}, \emph{23}(2), 242--266.

\leavevmode\hypertarget{ref-newman2005measure}{}%
Newman, M. E. (2005). A measure of betweenness centrality based on
random walks. \emph{Social Networks}, \emph{27}(1), 39--54.

\leavevmode\hypertarget{ref-organ_organizational_1988}{}%
Organ, D. W. (1988). \emph{Organizational citizenship behavior: The good
soldier syndrome.} Lexington Books/DC Heath and Com.

\leavevmode\hypertarget{ref-organ_organizational_2005}{}%
Organ, D. W., Podsakoff, P. M., \& MacKenzie, S. B. (2005).
\emph{Organizational citizenship behavior: Its nature, antecedents, and
consequences}. Sage Publications.

\leavevmode\hypertarget{ref-organ_meta-analytic_1995}{}%
Organ, D. W., \& Ryan, K. (1995). A meta-analytic review of attitudinal
and dispositional predictors of organizational citizenship behavior.
\emph{Personnel Psychology}, \emph{48}(4), 775--802.

\leavevmode\hypertarget{ref-penner_beyond_1997}{}%
Penner, L. A., Midili, A. R., \& Kegelmeyer, J. (1997). Beyond Job
Attitudes: A Personality and Social Psychology Perspective on the Causes
of Organizational Citizenship Behavior. \emph{Human Performance},
\emph{10}(2), 111--131.
doi:\href{https://doi.org/10.1207/s15327043hup1002_4}{10.1207/s15327043hup1002\_4}

\leavevmode\hypertarget{ref-podsakoff_individual-and_2009}{}%
Podsakoff, N. P., Whiting, S. W., Podsakoff, P. M., \& Blume, B. D.
(2009). Individual-and organizational-level consequences of
organizational citizenship behaviors: A meta-analysis. \emph{Journal of
Applied Psychology}, \emph{94}(1), 122.

\leavevmode\hypertarget{ref-podsakoff_impact_1997}{}%
Podsakoff, P. M., \& MacKenzie, S. B. (1997). Impact of organizational
citizenship behavior on organizational performance: A review and
suggestion for future research. \emph{Human Performance}, \emph{10}(2),
133--151.

\leavevmode\hypertarget{ref-podsakoff_organizational_2000}{}%
Podsakoff, P. M., MacKenzie, S. B., Paine, J. B., \& Bachrach, D. G.
(2000). Organizational citizenship behaviors: A critical review of the
theoretical and empirical literature and suggestions for future
research. \emph{Journal of Management}, \emph{26}(3), 513--563.

\leavevmode\hypertarget{ref-podsakoff_oxford_2018}{}%
Podsakoff, P. M., MacKenzie, S. B., \& Podsakoff, N. P. (2018).
\emph{The Oxford handbook of organizational citizenship behavior}.
Oxford University Press.

\leavevmode\hypertarget{ref-polson2012good}{}%
Polson, N. G., \& Scott, J. G. (2012). Good, great, or lucky? Screening
for firms with sustained superior performance using heavy-tailed priors.
\emph{The Annals of Applied Statistics}, \emph{6}(1), 161--185.

\leavevmode\hypertarget{ref-randic1980random}{}%
Randić, M. (1980). Random walks and their diagnostic value for
characterization of atomic environment. \emph{Journal of Computational
Chemistry}, \emph{1}(4), 386--399.

\leavevmode\hypertarget{ref-schreurs2012weekly}{}%
Schreurs, B. H., Hetty van Emmerik, I., Günter, H., \& Germeys, F.
(2012). A weekly diary study on the buffering role of social support in
the relationship between job insecurity and employee performance.
\emph{Human Resource Management}, \emph{51}(2), 259--279.

\leavevmode\hypertarget{ref-shang2018note}{}%
Shang, Y. (2018). A note on the h index in random networks. \emph{The
Journal of Mathematical Sociology}, \emph{42}(2), 77--82.

\leavevmode\hypertarget{ref-simon_rational_1956}{}%
Simon, H. A. (1956). Rational choice and the structure of the
environment. \emph{Psychological Review}, \emph{63}(2), 129.

\leavevmode\hypertarget{ref-simon_what_1992}{}%
Simon, H. A. (1992). What is an ``explanation'' of behavior?
\emph{Psychological Science}, \emph{3}(3), 150--161.

\leavevmode\hypertarget{ref-sims2014hierarchical}{}%
Sims, D. W., Reynolds, A. M., Humphries, N. E., Southall, E. J.,
Wearmouth, V. J., Metcalfe, B., \& Twitchett, R. J. (2014). Hierarchical
random walks in trace fossils and the origin of optimal search behavior.
\emph{Proceedings of the National Academy of Sciences}, \emph{111}(30),
11073--11078.

\leavevmode\hypertarget{ref-smith1983organizational}{}%
Smith, C., Organ, D. W., \& Near, J. P. (1983). Organizational
citizenship behavior: Its nature and antecedents. \emph{Journal of
Applied Psychology}, \emph{68}(4), 653.

\leavevmode\hypertarget{ref-spence_understanding_2011}{}%
Spence, J. R., Ferris, D. L., Brown, D. J., \& Heller, D. (2011).
Understanding daily citizenship behaviors: A social comparison
perspective. \emph{Journal of Organizational Behavior}, \emph{32}(4),
547--571. doi:\href{https://doi.org/10.1002/job.738}{10.1002/job.738}

\leavevmode\hypertarget{ref-stamovlasis2003complexity}{}%
Stamovlasis, D., \& Tsaparlis, G. (2003). A complexity theory model in
science education problem solving: Random walks for working memory and
mental capacity. \emph{Nonlinear Dynamics, Psychology, and Life
Sciences}, \emph{7}(3), 221--244.

\leavevmode\hypertarget{ref-stewart2007exploring}{}%
Stewart, G. L., \& Nandkeolyar, A. K. (2007). Exploring how constraints
created by other people influence intraindividual variation in objective
performance measures. \emph{Journal of Applied Psychology},
\emph{92}(4), 1149.

\leavevmode\hypertarget{ref-van1998helping}{}%
Van Dyne, L., \& LePine, J. A. (1998). Helping and voice extra-role
behaviors: Evidence of construct and predictive validity. \emph{Academy
of Management Journal}, \emph{41}(1), 108--119.

\leavevmode\hypertarget{ref-van2000effects}{}%
Van Scotter, J., Motowidlo, S. J., \& Cross, T. C. (2000). Effects of
task performance and contextual performance on systemic rewards.
\emph{Journal of Applied Psychology}, \emph{85}(4), 526.

\leavevmode\hypertarget{ref-wang2019s}{}%
Wang, L., Law, K. S., Zhang, M. J., Li, Y. N., \& Liang, Y. (2019). It's
mine! Psychological ownership of one's job explains positive and
negative workplace outcomes of job engagement. \emph{Journal of Applied
Psychology}, \emph{104}(2), 229.

\leavevmode\hypertarget{ref-wu2012role}{}%
Wu, C.-h., \& Parker, S. K. (2012). The role of attachment styles in
shaping proactive behaviour: An intra-individual analysis. \emph{Journal
of Occupational and Organizational Psychology}, \emph{85}(3), 523--530.


\end{document}
